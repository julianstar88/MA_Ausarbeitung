\chapter{Ausblick}
\label{kap:ausblick}

Obwohl die Kopplungschemie über \ac{NHS}/\ac{EDC} standardmäßig in einem engen pH-Bereich von 7 bis 9 stattfindet \cite{Hayworth,Fischer.2010,Nojima.2009}, konnte durch die spezielle Charakteristik der Force-Clamp-Versuche der pH-Bereich auf 4,5 bis 9 erweitert werden. Dabei werden Abtastspitze und Substrat auf eine Kraft von $250~pN$ zusammengebracht und über einen Zeitraum von $3~s$ ruhen gelassen (2. Segment). Somit waren alle Reaktionspartner in räumlicher Nähe zueinander und hatten genügend Zeit miteinander zu reagieren. Um die störenden Mehrfachanbindungen eines CMA-Moleküls an Abtastspitze und Substrat zu minimieren stellen die Haltekraft und Ruhezeit im 2. Segment somit die wichtigsten Parameter dar. In wie fern bei niedrigeren pH-Werten als 4,5 gemessen werden kann muss erst getestet werden. Fischer et al. \cite{Fischer.2010} spricht von einem pH-Wert von 3,5 als Grenze. Wobei diese Grenze für Force-Clamp-Experimente nicht zwingend gelten muss. Wie die erhaltenen Messergebnisse bei pH 4,5 zeigen, sollte durch herkömmlichen Kopplungsprotokolle keine nennenswerte Ausbildung von Amidbindungen stattfinden. Bei den durchgeführten Force-Clamp-Experimenten war die Ausbeute von Clampereignisse im niedrigen pH-Bereich am höchsten (vgl. \abb~\ref{fig:general_curve_composition} und \ref{fig:discarded_curve_composition}). Somit wäre die Messung der Amidbindung unterhalb von pH 3,5 mittels einer \ac{EDC}/\ac{NHS}-Kopplung möglich. Bei einem pH-Wert von 9 liegt die mittlere Lebensdauer der NHS-Ester bei wenigen Minuten \cite{Hayworth}. Dies macht die Überprüfung der Arbeitshypothese bei einem pH-Wert von über 9 schwierig. Es wäre denkbar für solche pH-Werte andere Methoden zur Bildung der Amidbindung zu verwenden. Ein Beispiel hierfür wäre die Kopplung von primären Aminen mit Carbonsäurehalogeniden wie sie in \cite{Montalbetti.2005} beschrieben wird.