\chapter{Fazit}
\label{kap:Fazit}

Mehrfachanbindungen eines einzelnen \ac{CMA}-Moleküls an Substrat und \spitze~(s. \abb~\ref{fig:multiple_bonding}) konnten den langsamen Prozess bei allen pH-Werten erklären. Durch räumliche Nähe der \spitze~zum Substrat (und damit Amino- und \carboxys) sowie die Wartezeit im 2. Segment führten zu einer vermehrten Bildung von \amide~bei allen pH-Werten. Im Vergleich von pH 8 zu pH 7,4 zeigt sich eine stärkere Bildung von \amide, dass weniger Clampereignisse im schnellen Prozess zu finden waren. Die Konkurrenzreaktion der \ch{OH-}-Ionen mit den \ac{NHS}-Estern bei pH 8 sorgte jedoch dafür, dass ausreichend Clampereignisse zur Ermittlung von $k_1^8$ in den schnellen Prozess fielen.\\

Bei pH 4,5 stand der Ausbildung von \amide~die protonierte Form der \aminos~entgegen. Die Protonierung der \aminos~läuft als Gleichgewichtsreaktion ab \cite[189]{Latscha.2016}. Im sauren Milieu liegt das Gleichgewicht auf der rechten Seite. Die Carbonylgruppe der \ac{NHS}-Ester liegt ebenfalls protoniert vor. Ein nukleophiler Angriff nicht protonierter \aminos~kann deshalb viel schneller erfolgen. Zusammen mit der hohen lokalen Konzentration aller Reaktionspartner und ausreichend Reaktionszeit im 2. Segment, konnten trotz des geringen Anteils an \aminos~mit freiem Elektronenpaar, im sehr effektiv \amide~gebildet werden. Durch einzelne, mehrfach angebundene \spacer~waren daher deutlich mehr Clampereignisse im langsamen Prozess vorhanden.\\

Die Ergebnisse stützten somit nicht die Arbeitshypothese, die Hydrolyse der Amidbindung verläuft mit steigendem pH-Wert langsamer. Zumindest bei den pH-Werten 7,4 und 8 verhielt sich die Hydrolysegeschwindigkeit der Amidbindung unter Einfluss einer äußeren Kraft wie in \cite{Smith.1998,Bundgaard.1991,Song.2000} beschrieben, jedoch um den Faktor $10^9$ beschleunigt. Bei einem pH-Wert von 4,5 konnte über die einzelne Amidbindung keine Aussage getroffen werden. Hier wurden hauptsächlich mehrere, hintereinander brechende Amidbindungen gemessen. Diese hintereinander abreißenden Bindungen konnten aufgrund des vorliegenden Messaufbaus nicht voneinander unterschieden werden. Wurden die langsamen Prozesse bei pH 7,4 und 4,5 miteinander vergleichen($k_2^{7,4} = 0,06~s^{-1}$ bzw. $k_2^{4,5} = 0,08~s^{-1}$), konnte auch hier eine Beschleunigung der Hydrolysegeschwindigkeit der Amidbindung festgestellt werden.



