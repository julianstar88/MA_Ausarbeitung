%%% Dokumentenklasse
\documentclass[10pt,a4paper,draft=false]{scrreprt}

%%% Kopf & Fußzeile
\usepackage[
automark,			%automatische Aktualisierung des Kolumnentitels in der Kopfzeile
headsepline,		%Strich UNTER der Kopfzeile
footsepline			%Strich ÜBER der Fußzeile
]{scrpage2}

%%% Sprach Pakete
\usepackage[ngerman]{babel}
%\usepackage[ansinew]{inputenc} Falls es Probleme mit ä,ö,ü ... gibt, sollte das Packet in dieser Zeile wieder aktiviert werden 
\usepackage[utf8]{inputenc} % wenn der Kommentar aus der oberen Zeile entfehrnt wird, muss diese Zeile kommentiert werden
\usepackage[babel, german=quotes]{csquotes}
\usepackage[T1]{fontenc}

%%% Style Packete
\usepackage{geometry}
\usepackage{graphicx}
\usepackage{textcomp}
\usepackage{float}
\usepackage{longtable} 	
\usepackage{multirow,array,booktabs} 
\usepackage{tabularx}
\usepackage[bf, flushleft]{caption}[2002/08/03]
\usepackage{subcaption}
\usepackage{threeparttable}
\usepackage{upgreek} % Nicht kursive Griechische Buchstaben einbinden
\usepackage{textgreek} % For greek letters in the text

%%% Mathematik Packete
\usepackage{amsfonts}
\usepackage{amssymb}
\usepackage{setspace}
\usepackage{amsmath}
\usepackage[amssymb]{SIunits} % Einbinden von SI - Einheiten

%%%Chemie Packete
\usepackage{chemmacros}
\usepackage{chemformula}
\usepackage{chemfig}

%%% Abkürzungen
\usepackage[printonlyused,withpage]{acronym}

%%% Spaltentypen in Tabellen (mit dem Packet "tabularx")
\usepackage{array}
\newcolumntype{L}{>{\raggedright\arraybackslash}X} % linksbündig mit Breitenangabe
\newcolumntype{C}{>{\centering\arraybackslash}X} % zentriert mit Breitenangabe
\newcolumntype{R}{>{\raggedleft\arraybackslash}X} % rechtsbündig mit Breitenangabe

% Hyperref: Klickbare Verweise in PDF-Dokumenten bei Verwendung von pdftex.
%           Einstellbare Optionen für die PDF-Dokumenteigenschaften:
%           - pdftitle:  Titel des Dokuments
%           - pdfauthor: Autor der Arbeit
%			- linkbordercolor: Farbe der Boxen um den Link (=interner Link)
%			- colorlinks: Text des Links (=interner Link) farbig machen
%			- linkcolor: Farbe des Links (=interner Link)
%			- linktocpage: boxen um die Seitenzahlen, statt den Text
%           ACHTUNG: Leerzeichen sind in den Optionen nicht zulässig. Am
%           besten geschützte Leerzeichen (~) benutzen.
\usepackage[linkbordercolor=white,colorlinks=true,linkcolor=blue]{hyperref} 

%--------------------Tabellenformat--------------------
%\begin{table}[H]
%	\begin{threeparttable}
%	\caption[text]{Text}
%		\begin{tabularx}{\textwidth}{X X X X}
%		
%		\end{tabularx}
%		\begin{tablenotes}
%		\item[#] note...
%		\end{tablenotes}
%	\label{text}
%	\end{threeparttable}
%\end{table}
%--------------------Tabellenformat--------------------


%%%% Bibliografie
%\usepackage[
%backend=biber,
%sorting=none,
%style=numeric-comp,    		   % Zitierstil
%isbn=false,     % isbn nicht anzeigen, gleiches geht mit nahezu allen anderen Feldern
%url=false,		%url wird nicht angezeigt
%doi=true,		%doi wird angezeigt
%eprint=false,	%eprint wird nicht angezeigt
%pagetracker=true,          % ebd. bei wiederholten Angaben (false=ausgeschaltet, page=Seite, spread=Doppelseite, true=automatisch)
%ibidtracker=true,
%maxbibnames=50,            % maximale Namen, die im Literaturverzeichnis angezeigt werden (ich wollte alle)
%maxcitenames=3,            % maximale Namen, die im Text angezeigt werden, ab 4 wird u.a. nach den ersten Autor angezeigt
%autocite=inline,           % regelt Aussehen für \autocite (inline=\parancite)
%block=space,               % kleiner horizontaler Platz zwischen den Feldern
%%backref=true,              % Seiten anzeigen, auf denen die Referenz vorkommt
%backrefstyle=three+,       % fasst Seiten zusammen, z.B. S. 2f, 6ff, 7-10
%date=short,                % Datumsformat
%]{biblatex}
%\setlength{\bibitemsep}{1em}     % Abstand zwischen den Literaturangaben
%\setlength{\bibhang}{2em}        % Einzug nach jeweils erster Zeile

%--------------------------------------------------

% %% Bibliographie laden
%\bibliography{Pfad} 

%%% Kapitel einbinden
%\includeonly{
%Liste der Kapitel
%}
%--------------------------------------------------

%\author{Julian Blaser}
%\title{\Huge \textbf{Titel}}
%\date{\today}

%------------Fancy Header------------------
%\pagestyle{fancy}
%\renewcommand{\sectionmark}[1]{\markboth{\emph{#1}}{}}
%\fancyhf{}

%\fancyhead[LE,RO]{\textbf{\thepage}}
%\fancyhead[LO,RE]{\textbf{\leftmark}}
%\fancyfoot[LE,RO]{\author{Julian Blaser}}
%\fancyfoot[LO,RE]{\date{\today}}
%\renewcommand{\headrulewidth}{0.5pt}
%\renewcommand{\footrulewidth}{0.5pt}
%----------Ende Fancy Header---------------


%%%% ------------Kopf&Fußzeile KOMA Skript------------------
%\pagestyle{scrheadings}
%
%\ihead{\raisebox{1.mm}{\headmark}}
%\ohead{\raisebox{1.mm}{\pagemark}}
%\chead{}
%\ofoot{\raisebox{-1.5mm}{Julian Blaser}}
%\ifoot{\raisebox{-1.5mm}{Hochschule München}}
%\cfoot{}
%
%\setheadsepline{.5pt}
%\setfootsepline{.5pt}
%
%%%% ------------Kopf&Fußzeile KOMA Skript------------------

\setlength{\parindent}{0em} % Automatisches Einrücken nach Absatz verhindern

\begin{document}
\setchemfig{angle increment=30, atom style={scale=0.5},  fixed length=true, arrow coeff=0.8, bond join=true}
\renewcommand*\printatom[1]{\rotatebox{-0}{\ensuremath{\mathrm{#1}}}}

%%%% Start Makros %%%%


%% Prototyp CMA
% 1?[a](-[2,1.5]6(-[7,0.75]!\OH)-[-0.6,1.75]5?[close]-[4,0.75]!\OR)-[1]2(-[-2,0.75]-[-3,0.75]!\OR)-[-0.6,1.75]3-[2,1.5]4?[close] %% Version mit Hook 
% 1(-[1.6,1.5]6(-[7,0.75]!\OH)-[-0.8,1.75]5(-[4,0.75]!\OR)-[1,0.8])-[1]2(-[-2,0.75]-[-3,0.75]!\OR)-[-0.8,1.75]3-[1.6,1.5]4 %% Version ohne Hook 
% 1(-[:48,1.5]6(-[:210,0.75]!\OH)-[:-24,1.75]5(-[:120,0.75]!\OR)-[:30,0.8])-[:30]2(-[:-60,0.75]-[:-120,0.75]!\OR)-[:-24,1.75]3-[:48,1.5]4 %% Version mit absoluten Winkeln
% 1(-[::70,1.5]6(-[::140,0.75]!\OH)-[::-80,1.75]5(-[::130,0.75]!\OR)-[::40,0.9])-[::30]2(-[::-90,0.75]-[::-30,0.75]!\OH)-[::-40,1.75]3-[::80,1.5]4 %% Version mit relativen Winkeln
% 1-[::30,0.75]-[::30,0.75](-[::150]-[::-140,1.5](-[::140,0.75]!\OH)-[::-80,1.75](-[::130,0.75]!\OR)-[::40,1.2])-[::-70,1.75]O-[::80,1.5] %% Die #1 ist an einem Ast mit Rest; damit kann die Kopplung von Allylamin zu CMA dargestellt werden

%% Prototyp CMAr (rotated)
% 1?[a](-[-1]2(-[2,0.75]-[3,0.75]!\OR)-[0.8,1.75]3?[close])-[-1.6,1.5]6(-[5,0.75]!\OH)-[0.8,1.75]5(-[8,0.75]!\OR)-[-1]4?[close] %% Version mit Hook 
% 1(-[-1]2(-[2,0.75]-[3,0.75]!\OR)-[0.8,1.75]3-[-1.6,1.4])-[-1.6,1.5]6(-[5,0.75]!\OH)-[0.8,1.75]5(-[8,0.75]!\OR)-[-1]4 %% Version ohne Hook 
% 1(-[:-30]2(-[:60,0.75]-[:90,0.75]!\OR)-[:24,1.75]3-[:-48,1.4])-[:-48,1.5]6(-[:150,0.75]!\OH)-[:24,1.75]5(-[:240,0.75]!\OR)-[:-30]4 %% Version mit absoluten Winkeln
% (-[::-30](-[::90,0.75]-[::30,0.75]!\OR)-[::40,1.75]O-[::-80,1.5])-[::-70,1.5](-[::240,0.75]!\OH)-[::80,1.75](-[::-120,0.75]!\OR)-[::-40,1.2] %% Version mit relativen Winkeln

%% Makros für Polymerklammern (für Details siehe Dokumentation des Chemfig Paketes)
\newcommand\setpolymerdelim[2]{\def\delimleft{#1}\def\delimright{#2}}
\def\makebraces[#1,#2]#3#4#5{%
	\edef\delimhalfdim{\the\dimexpr(#1+#2)/2}%
	\edef\delimvshift{\the\dimexpr(#1-#2)/2}%
	\chemmove{%
		\node[at=(#4),yshift=(\delimvshift)]{$\left\delimleft\vrule height\delimhalfdim depth\delimhalfdim width0pt\right.$};
		\node[at=(#5),yshift=(\delimvshift)]{$\left.\vrule height\delimhalfdim depth\delimhalfdim width0pt\right\delimright_{\rlap{$\scriptstyle#3$}}$};
	} % \chemmove
} % \def\delimleft\makebraces

% Beispiel: Polyethylen
%
% \setpolymerdelim[]
% \chemfig{\vphantom{CH_2}-[@{op,.75}]CH_2-CH_2-[@{cl,0.25}]} 
% \makebraces[5pt,5pt]{\!n}{op}{cl}

%% Diese Befehle sorgen dafür dass funktionelle Gruppen immer am O bzw. C angesetzt werden
\definesubmol\OH[HO]{OH}
\definesubmol\OR[RO]{OR}

%% CMA moleküle (\CMA - normales CMA und \CMAr - rotieretes CMA)
\definesubmol\CMA{
	(-[::70,1.5](-[::140,0.75]!\OH)-[::-80,1.75](-[::130,0.75]!\OR)-[::40,1.2])-[::30](-[::-90,0.75]-[::-30,0.75]!\OR)-[::-40,1.75]O-[::80,1.5]
}


\definesubmol\CMAVerbindung{
	O-[::60]-[::30,0.75](-[::150](-[::-60]...)-[::-140,1.5](-[::140,0.75]!\OH)-[::-80,1.75](-[::130,0.75]!\OR)-[::40,1.2])-[::-70,1.75]O-[::80,1.5]
}

\definesubmol\CMAr{
	(-[::-30](-[::90,0.75]-[::30,0.75]!\OR-[@{left,0.25}::-30,,,,draw=none])-[::40,1.75]O-[::-80,1.5])-[::-70,1.5](-[::240,0.75]!\OH)-[::80,1.75](-[::-120,0.75]!\OR-[@{right,0.25}::10,,,,draw=none])-[::-40,1.2]
}

\definesubmol\CMArVerbindung{
	(-[::-30](-[::90,0.75]!\OR)-[::40,1.75]O-[::-80,1.5])-[::-70,1.5](-[::240,0.75]!\OH)-[::80,1.75](-[::-40,1.2]-[::60]...)-[::-70,0.75]-[::30]O
}

%% Allylamin Reste
%DLC-Substrat zu Allylamin
\definesubmol\AllylaminOF{
	C(-[::90,,,,thick]C-[::0,,,,thick]C)(-[::-90,,,,thick]C-[::0,,,,thick]C)-[::-30]-[::60]-[::-60]\textcolor{red}{{\textbf N}}(-[::-60,0.75,,,red,thick]\textcolor{red}{{\textbf H}})-[::60,,,,,red,thick](=[::60,0.75,,,red]\textcolor{red}{{\textbf O}})
}

% Allylamin zu AFM-Spitze
\definesubmol\AllylaminSP{
	(=[::60,0.75,,,red]\textcolor{red}{{\textbf O}})-[::-60,,,,red,thick]\textcolor{red}{{\textbf N}}(-[::-60,0.75,,,red,thick]\textcolor{red}{{\textbf H}})-[::60]-[::-60]-[::60]C(-[::60,,,,thick]C-[::0,,,,thick]C)(-[::-120,,,,thick]C-[::0,,,,thick]C)
}

%%%% Ende Makros %%%%

%\scalebox{1}{\begin{minipage}{\textwidth}%Empfohlene Umbegung, für gute Skalierbarkeit
%		\setpolymerdelim[]
%		\setchemfig{atom style={rotate=0,scale=0.5}}
%		\chemfig{[:0]!\AllylaminOF-[::-60]-[::30,0.001,,,,draw=none]!\CMA-[@{leftOut,0.5}::-100]O-[::60]-[::-30,0.001,,,,draw=none]!\CMAr-[::60]O-[::-60]-[::30,0.001,,,,draw=none]!\CMA-[@{rightOut,0.5}::-100]O-[::60]-[::-30,0.001,,,draw=none]!\CMAr-[::60]!\AllylaminSP}
%		%\makebraces[45pt,30pt]{\!n}{left}{right}
%		\makebraces[30pt,50pt]{\!n}{leftOut}{rightOut}
%	\end{minipage}
%}

\scalebox{1}{\begin{minipage}{\textwidth}%Empfohlene Umbegung, für gute Skalierbarkeit
		\setpolymerdelim[]
		\setchemfig{atom style={rotate=0,scale=0.5}}
		\chemfig{[:0]!\AllylaminOF-[::-60]!\CMAVerbindung-[@{leftOut,0.5}::-100]O-[::60]-[::-30,0.001,,,,draw=none]!\CMAr-[::60]O-[::-60]-[::30,0.001,,,,draw=none]!\CMA-[@{rightOut,0.5}::-100]O-[::60]-[::-30,0.001,,,draw=none]!\CMArVerbindung-[::60]!\AllylaminSP}
		%\makebraces[45pt,30pt]{\!n}{left}{right}
		\makebraces[30pt,50pt]{\!n}{leftOut}{rightOut}
	\end{minipage}
}


\end{document}
