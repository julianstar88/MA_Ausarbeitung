%%% Dokumentenklasse
\documentclass[10pt,a4paper,draft=false]{scrreprt}
\KOMAoption{toc}{listof,bib}

%%% Kopf & Fußzeile
\usepackage[
automark,			%automatische Aktualisierung des Kolumnentitels in der Kopfzeile
headsepline,		%Strich UNTER der Kopfzeile
footsepline			%Strich ÜBER der Fußzeile
]{scrpage2}

%%% Sprach Pakete
\usepackage[ngerman]{babel}
%\usepackage[ansinew]{inputenc} Falls es Probleme mit ä,ö,ü ... gibt, sollte das Packet in dieser Zeile wieder aktiviert werden 
\usepackage[utf8]{inputenc} % wenn der Kommentar aus der oberen Zeile entfehrnt wird, muss diese Zeile kommentiert werden
\usepackage[babel, german=quotes]{csquotes}
\usepackage[T1]{fontenc}
\usepackage{helvet}

%%% Style Packete
\usepackage{geometry}
\usepackage{graphicx}
\usepackage{textcomp}
\usepackage{float}
\usepackage{longtable} 	
\usepackage{tabularx}
\usepackage{ltablex} % kombiniert longtable und tabularx; benötigt den befehl \keepXColumns direkt vor \begin{tabularx} damit der X-Character der spalten erhalten bleibt
\usepackage{multirow,array,booktabs} 
\usepackage[bf, flushleft]{caption}[2002/08/03]
\usepackage{subcaption}
\usepackage{threeparttable}
\usepackage{upgreek} % Nicht kursive Griechische Buchstaben einbinden
\usepackage{textgreek} % For greek letters in the text

% Hyperref: Klickbare Verweise in PDF-Dokumenten bei Verwendung von pdftex.
%           Einstellbare Optionen für die PDF-Dokumenteigenschaften:
%           - pdftitle:  Titel des Dokuments
%           - pdfauthor: Autor der Arbeit
%           ACHTUNG: Leerzeichen sind in den Optionen nicht zulässig. Am
%           besten geschützte Leerzeichen (~) benutzen.
\usepackage[pdftitle={Titel},pdfauthor={Autor},bookmarksopen=true,bookmarksopenlevel=2]{hyperref}

%%% Mathematik Packete
\usepackage{amsfonts}
\usepackage{amssymb}
\usepackage{setspace}
\usepackage{amsmath}
\usepackage[amssymb]{SIunits} % Einbinden von SI - Einheiten

%Chemie Packete
\usepackage{chemmacros}
\usepackage{chemformula}
\usepackage{chemfig}

%%% Abkürzungen
\usepackage[printonlyused,withpage]{acronym}

%%% Spaltentypen in Tabellen (mit dem Packet "tabularx")
\usepackage{array}
\newcolumntype{L}{>{\raggedright\arraybackslash}X} % linksbündig mit Breitenangabe
\newcolumntype{C}{>{\centering\arraybackslash}X} % zentriert mit Breitenangabe
\newcolumntype{R}{>{\raggedleft\arraybackslash}X} % rechtsbündig mit Breitenangabe

%%% Bibliografie
\usepackage[
backend=biber,
sorting=none,
style=numeric-comp,    		   % Zitierstil
isbn=false,     % isbn nicht anzeigen, gleiches geht mit nahezu allen anderen Feldern
url=false,		%url wird nicht angezeigt
doi=true,		%doi wird angezeigt
eprint=false,	%eprint wird nicht angezeigt
pagetracker=true,          % ebd. bei wiederholten Angaben (false=ausgeschaltet, page=Seite, spread=Doppelseite, true=automatisch)
ibidtracker=true,
maxbibnames=50,            % maximale Namen, die im Literaturverzeichnis angezeigt werden (ich wollte alle)
maxcitenames=3,            % maximale Namen, die im Text angezeigt werden, ab 4 wird u.a. nach den ersten Autor angezeigt
autocite=inline,           % regelt Aussehen für \autocite (inline=\parancite)
block=space,               % kleiner horizontaler Platz zwischen den Feldern
backref=true,              % Seiten anzeigen, auf denen die Referenz vorkommt
backrefstyle=three+,       % fasst Seiten zusammen, z.B. S. 2f, 6ff, 7-10
date=short,                % Datumsformat
]{biblatex}
\setlength{\bibitemsep}{1em}     % Abstand zwischen den Literaturangaben
\setlength{\bibhang}{2em}        % Einzug nach jeweils erster Zeile

%--------------------------------------------------

% %% Bibliographie laden
%\bibliography{Pfad} 

%%% Kapitel einbinden
%\includeonly{
%Liste der Kapitel
%}
%--------------------------------------------------

\author{Julian Blaser}
\title{\Huge \textbf{Titel}}
\date{\today}

%------------Fancy Header------------------
%\pagestyle{fancy}
%\renewcommand{\sectionmark}[1]{\markboth{\emph{#1}}{}}
%\fancyhf{}

%\fancyhead[LE,RO]{\textbf{\thepage}}
%\fancyhead[LO,RE]{\textbf{\leftmark}}
%\fancyfoot[LE,RO]{\author{Julian Blaser}}
%\fancyfoot[LO,RE]{\date{\today}}
%\renewcommand{\headrulewidth}{0.5pt}
%\renewcommand{\footrulewidth}{0.5pt}
%----------Ende Fancy Header---------------


%%% ------------Kopf&Fußzeile KOMA Skript------------------
\pagestyle{scrheadings}

\ihead{\raisebox{1.mm}{\headmark}}
\ohead{\raisebox{1.mm}{\pagemark}}
\chead{}
\ofoot{\raisebox{-1.5mm}{Julian Blaser}}
\ifoot{\raisebox{-1.5mm}{Hochschule München}}
\cfoot{}

\setheadsepline{.5pt}
\setfootsepline{.5pt}

%%% ------------Kopf&Fußzeile KOMA Skript------------------

\setlength{\parindent}{0em} % Automatisches Einrücken nach Absatz verhindern

\renewcommand{\familydefault}{\sfdefault}

\begin{document}


%%% Aimidbindung
\begin{center}
	\setchemfig{angle increment=30}
	\definesubmol{carbonyl}{\textcolor{red}{C}=[3,0.75,,,red]\textcolor{red}{O}}
	\chemfig{R_1-[-1]\textcolor{red}{C}-[1]C(!{carbonyl})-[-1]NH-[1]R_2}
	\bigskip
\end{center}

\vspace{3cm}

%%% Raktivität der Carbonsäurederivate
\begin{center}
	\setchemfig{scheme debug = false}
	\setchemfig{angle increment=30}
	\definesubmol{rest}{R-[,0.75]C(=[2,0.75]O)-[-2,0.75]}
	\schemestart
		\chemname{
			\chemfig{!{rest}OH}
		}{Carbonsäure}
		\arrow(.base east--.base west){0}[,0]<\arrow(.base east--.base west){0}[,0.3]
		\chemname{
			\chemfig{!{rest}NH_2}
		}{-amid}
		\arrow(.base east--.base west){0}[,0]<\arrow(.base east--.base west){0}[,0.3]
		\chemname{
			\chemfig{!{rest}OR}
		}{-ester}
		\arrow(.base east--.base west){0}[,0]<\arrow(.base east--.base west){0}[,0.3]
		\chemname{
			\chemfig{!{rest}SR}
		}{-thioester}
		\arrow(.base east--.base west){0}[,0]<\arrow(.base east--.base west){0}[,0.3]
		\chemname{
			\chemfig{R-[,0.75]C(=[2,0.75]O)-[-1,0.75]O-[1,0.75]C(=[4,0.75]O)-[,0.75]R}
		}{-anhydrid}
		\arrow(.base east--.base west){0}[,0.3]<\arrow(.base east--.base west){0}[,0.3]
		\chemname{
			\chemfig{!{rest}Cl}
		}{-chlorid}
	\schemestop
\end{center}

\vspace{3cm}

%%% Komplette NHS/EDC-Chemie inklusive makros

%% Makros für die Bestandteile der Kopplungschemie

% EDC
\newcommand{\EDC}{	
	\chemfig{H_3C-[:30,.75]-[:-30,.75]N=[:30,.75]C=[:30,.75]N-[:-30,.75]-[:30,.75]-[:-30,.75]-[:30,.75]\chembelow{N\rlap{${}^+$}}{H}(-[:90]CH_3)(-[:-30]CH_3)
		(-[:30,0.6,,,draw=none]Cl\rlap{${}^-$})}
}

% Isourea
\newcommand{\OAcylisourea}{
	\chemfig{
		[:90,.75]H_3C-[:30]-[:-30]N-[:30](-O-[:30](-\textcolor{red}{\textbf R_1})=[:-30]O)=[:-30]N-[:30]-[:-30]-[:30]-[:-30]\chembelow{N\rlap{${}^+$}}{H}(-[:90]CH_3)(-[:-30]CH_3)
	}
}

% Rest für den NHS-Ester
\definesubmol{REster}{\textcolor{red}{\textbf R_1}-[::30,.75](=[::60,.75]O)-[::-60,.75]O}

% NHS-Ester
\newcommand{\Ester}{
	\chemfig{
		[,.75]!{REster}-[:30,1]N*5(-(=O)---(=O)-)
	}	
}

% NHS
\newcommand{\NHS}{
	\chemfig{
		[:30,.75]HO-[,1]N*5(-(=O)---(=O)-)
	}
}

% Amidbindung
\newcommand{\Amidbindung}{
	\chemfig{
		[:30,1]\textcolor{red}{\textbf R_1}-(=[:90,.75]O)-[:-30]N(-[:-90,.75]H)-\textcolor{red}{\textbf R_2}
	}	
}

% Primäres Amin
\newcommand{\PrimAmin}{
	\chemfig{
		[,1.25]H_2N-\textcolor{red}{\textbf R_2}
	}	
}

% Rest mit Carbonsäure
\newcommand{\Rest}{
	\chemfig{
		[,.75]\textcolor{red}{\textbf R_1}-[:30](=[:90]O)-[:-30]OH
	}
}


%% Ablaufschema der Kopplung

\begin{figure}[H]
	\begin{center}
		\scalebox{.75}{\begin{minipage}{\textwidth}
				\schemedebug{false}
				\schemestart
				\chemname{\Rest}{Carbonsäure} 
				\arrow(.south--.north west){-U>[*{0.south west}\chemname{\EDC}{EDC}][][][.5][]}[-90,4] 
				\chemname{\OAcylisourea}{O-Acylisourea Zwischenprodukt}
				\arrow{-U>[*{0.south}\chemname{\NHS}{NHS}][][][.5][]}[,3]
				\chemname{\Ester}{NHS Ester Zwischenprodukt}
				\arrow{-U>[*{0.north east}\chemname{\PrimAmin}{Primäres Amin}][*{0.east}\chemname{\NHS}{NHS}][][.3][60]}[90,4]
				\chemname{\Amidbindung}{Bildung der Amidbindung}
				\schemestop
			\end{minipage}
		}
	\end{center}
\end{figure}

\vspace{3cm}

%%% Ablaufschemata der Amidhydrolyse

%% Makros

% Wasser
\newcommand{\Wasser}{
	\setchemfig{angle increment = 30}
	\chemfig{H-[1,0.75]@{n1}\lewis{13,O}-[-1,0.75]H}
}

% OH-Ion
\newcommand{\OH}{
	\setchemfig{angle increment = 30}
	\chemfig{
	H@{oh}\chemabove{\lewis{026,O}}{\hspace{12pt}\rlap{-}}
	}
}


% Amidbindung allgemein (Amidbindung)
\renewcommand{\Amidbindung}{
	\chemfig{
		[:30,1]\textcolor{red}{\textbf R_1}-@{n2}(=[:90,.75]O)-[:-30]\chembelow{N}{H}-\textcolor{red}{\textbf R_2}
	}	
}

% Amidbindung mit addiertem Wasser und protoniert im neutralen (AmidbindungII)
\newcommand{\AmidbindungII}{
	\setchemfig{angle increment = 30}
	\chemfig{
	\textcolor{red}{\textbf R_1}-[1](-[-3.5,0.75]HO)(-[@{n3}3,0.75]@{n4}\lewis{024,O})-[@{n5}-1]@{n6}\chembelow{N}{\hspace{-5pt}\rlap{+}}H_2-[1]\textcolor{red}{\textbf R_2}
	}
}

% protonierte Amidbindung (AmidbindungIII)
\newcommand{\AmidbindungIII}{
	\setchemfig{angle increment = 30}
	\chemfig{
	\textcolor{red}{\textbf R_1}-[1](-[3,0.75]\lewis{24,O}H)(-[-3,0.2,,,draw = none]@{pa1}\hspace{-5pt}\rlap{+})-[-1]\chembelow{N}{H}-[1]\textcolor{red}{\textbf R_2}
	}
}

% Amidbindung mit addiertem Wasser im sauren (AmidbindungIV)
\newcommand{\AmidbindungIV}{
	\setchemfig{angle increment = 30}
	\chemfig{
	\textcolor{red}{\textbf R_1}-[1](-[3,0.75]\lewis{24,O}H)(-[-3.5,0.75]HO)-[-1]\chembelow{N}{H}-[1]\textcolor{red}{\textbf R_2}
	}
}

% Amidbindung-Wasser-addukt im sauren mit protoniertem Stickstoff (AmidbindungV)
\newcommand{\AmidbindungV}{
	\setchemfig{angle increment = 30}
	\chemfig{
	\textcolor{red}{\textbf R_1}-[1](-[@{av1}3,0.75]\lewis{24,O}-[@{av2},0.75]@{av3}H)(-[-3.5,0.75]HO)-[@{av4}-1]\chembelow{N}{\hspace{-5pt}\rlap{+}}H_2-[1]\textcolor{red}{\textbf R_2}
	}
}

% Amidbindung-OH-addukt im basischen (AmidbindungVI)
\newcommand{\AmidbindungVI}{
	\setchemfig{angle increment = 30}
	\chemfig{
	\textcolor{red}{\textbf R_1}-[1](-[3,0.75]\chemabove{\lewis{024,O}}{\hspace{10pt}\rlap{-}})(-[-3.5,0.75]HO)-[-1]\chembelow{N}{H}-[1]\textcolor{red}{\textbf R_2}
	}
}

% Amidbindung-OH-addukt mit protoniertem Stickstoff im basischen (AmibdingungVII)
\newcommand{\AmidbindungVII}{
	\setchemfig{angle increment = 30}
	\chemfig{
		\textcolor{red}{\textbf R_1}-[1](-[@{aoh1}3,0.75]@{aoh2}\chemabove{\lewis{024,O}}{\hspace{10pt}\rlap{-}})(-[-3.5,0.75]HO)-[@{aoh3}-1]@{aoh4}\chembelow{N}{\hspace{-5pt}\rlap{+}}H_2-[1]\textcolor{red}{\textbf R_2}
	}
}

% Carbonsäure
\newcommand{\Acid}{
	\setchemfig{angle increment = 30}
	\chemfig{
	\textcolor{red}{\textbf R_1}-[1](=[3,0.75]O)-[-1]OH
	}
}

% primäres Amin
\renewcommand{\PrimAmin}{
	\setchemfig{angle increment = 30}
	\chemfig{
	H_2\lewis{2,N}-\textcolor{red}{\textbf R_2}
	}
}

% protoniertes Amin
\newcommand{\PrimAminII}{
	\setchemfig{angle increment = 30}
	\chemfig{
	H_3\chemabove{N}{\hspace{-5pt}\rlap{+}}-\textcolor{red}{\textbf R_2}
	}
}

%% Hydrolyse im neutralen

\begin{figure}[H]
	\begin{center}
		\scalebox{0.75}{\begin{minipage}{\textwidth}
			\setchemfig{scheme debug = false}
			\schemestart
				\Wasser
				\hspace{1cm}
				\chemname{\Amidbindung}{N1}
				\chemmove[shorten <= 5pt, shorten >= 2pt]{\draw[dashed](n1)..controls +(north east:1.5cm) and +(south:2.75cm)..(n2);}
				\arrow{<=>}
				\chemname{\AmidbindungII}{N2}
				\chemmove[shorten <= 5pt, shorten >= 2pt]{\draw[dashed](n4)..controls +(east:0.5cm) and +(east:0.75cm)..(n3);
															\draw[dashed](n5)..controls +(north east:0.5cm) and +(north:0.5cm)..(n6);}
				\arrow{<=>}
				\chemname{\Acid}{N3} \+ \chemname{\PrimAmin}{N4}
			\schemestop
			\end{minipage}
		}
	\end{center}
\end{figure}

\vspace{3cm}

%% Hydrolyse im sauren

\begin{figure}[H]
	\begin{flushleft}
		\scalebox{0.75}{\begin{minipage}{\textwidth}
				\setchemfig{scheme debug = false}
				\schemestart
					\Wasser
					\hspace{0.1cm}
					\chemname{\AmidbindungIII}{S1}
					\chemmove[shorten <= 5pt, shorten >= 5pt]{\draw[dashed](n1)..controls +(north east:1.5cm) and +(240:2.75cm)..(pa1);}
					\arrow{<=>[-H$_{\mathrm{aq}}^{\tiny +}$][+H$_{\mathrm{aq}}^{\tiny +}$]}
					\chemname{\AmidbindungIV}{S2}
					\arrow{<=>[+H$_{\mathrm{aq}}^{\tiny +}$][-H$_{\mathrm{aq}}^{\tiny +}$]}
					\chemname{\AmidbindungV}{S3}
					\chemmove[shorten <= 2pt, shorten >= 2pt]{\draw[dashed](av2)..controls +(north:1.5cm) and +(west:1.5cm)..(av1);
																					\draw[dashed](av4)..controls +(350:0.75cm) and +(south east:0.75cm)..(av3);}
					\arrow{<=>}
					\chemname{\Acid}{S4} \+ \chemname{\PrimAminII}{S5}
				\schemestop
		\end{minipage}
	}
	\end{flushleft}
\end{figure}

\vspace{3cm}

% Hydrolyse im basischen

\begin{figure}[H]
	\begin{flushleft}
		\scalebox{0.75}{\begin{minipage}{\textwidth}
				\setchemfig{scheme debug = false}
				\schemestart
					\OH
					\hspace{0.5cm}
					\chemname{\Amidbindung}{B1}
					\chemmove[shorten <= 5pt, shorten >= 2pt]{\draw[dashed](oh)..controls +(330: 1cm) and +(260:1cm)..(n2);
					}
					\arrow{<=>}
					\chemname{\AmidbindungVI}{B2}
					\arrow{<=>[+H$_{\mathrm{aq}}^{\tiny +}$][-H$_{\mathrm{aq}}^{\tiny +}$]}
					\chemname{\AmidbindungVII}{B3}
					\chemmove[shorten <= 5pt, shorten >= 2pt]{\draw[dashed](aoh2)..controls +(east:0.5cm) and +(east:0.75cm)..(aoh1);
																					\draw[dashed](aoh3)..controls +(north east:0.5cm) and +(north:0.5cm)..(aoh4);
					}
					\arrow{<=>}
					\chemname{\Acid}{B4} \+ \chemname{\PrimAmin}{B5}
				\schemestop
			\end{minipage}
		}
	\end{flushleft}
\end{figure}


%%% Molekularer Aufbau der Experimente

%% Makros

% Polymerklammern (für Details siehe Dokumentation des Chemfig Paketes)
\newcommand\setpolymerdelim[2]{\def\delimleft{#1}\def\delimright{#2}}
\def\makebraces[#1,#2]#3#4#5{%
	\edef\delimhalfdim{\the\dimexpr(#1+#2)/2}%
	\edef\delimvshift{\the\dimexpr(#1-#2)/2}%
	\chemmove{%
		\node[at=(#4),yshift=(\delimvshift)]{$\left\delimleft\vrule height\delimhalfdim depth\delimhalfdim width0pt\right.$};
		\node[at=(#5),yshift=(\delimvshift)]{$\left.\vrule height\delimhalfdim depth\delimhalfdim width0pt\right\delimright_{\rlap{$\scriptstyle#3$}}$};
	} % \chemmove
} % \def\delimleft\makebraces

% Beispiel: Polyethylen
%
% \setpolymerdelim[]
% \chemfig{\vphantom{CH_2}-[@{op,.75}]CH_2-CH_2-[@{cl,0.25}]} 
% \makebraces[5pt,5pt]{\!n}{op}{cl}

% Diese Befehle sorgen dafür dass funktionelle Gruppen immer am O bzw. C angesetzt werden
\definesubmol\OH[HO]{OH}
\definesubmol\OR[RO]{OR}

% CMA moleküle (\CMA - normales CMA und \CMAr - rotieretes CMA)
\definesubmol\CMA{
	(-[::70,1.5](-[::140,0.75]!\OH)-[::-80,1.75](-[::130,0.75]!\OR)-[::40,1.2])-[::30](-[::-90,0.75]-[::-30,0.75]!\OR)-[::-40,1.75]O-[::80,1.5]
}


\definesubmol\CMAVerbindung{
	O-[::60]-[::30,0.75](-[::150](-[::-60]...)-[::-140,1.5](-[::140,0.75]!\OH)-[::-80,1.75](-[::130,0.75]!\OR)-[::40,1.2])-[::-70,1.75]O-[::80,1.5]
}

\definesubmol\CMAr{
	(-[::-30](-[::90,0.75]-[::30,0.75]!\OR-[@{left,0.25}::-30,,,,draw=none])-[::40,1.75]O-[::-80,1.5])-[::-70,1.5](-[::240,0.75]!\OH)-[::80,1.75](-[::-120,0.75]!\OR-[@{right,0.25}::10,,,,draw=none])-[::-40,1.2]
}

\definesubmol\CMArVerbindung{
	(-[::-30](-[::90,0.75]!\OR)-[::40,1.75]O-[::-80,1.5])-[::-70,1.5](-[::240,0.75]!\OH)-[::80,1.75](-[::-40,1.2]-[::60]...)-[::-70,0.75]-[::30]O
}

% Allylamin Reste

%DLC-Substrat zu Allylamin
\definesubmol\AllylaminOF{
	C(-[::90,,,,thick]C-[::0,,,,thick]C)(-[::-90,,,,thick]C-[::0,,,,thick]C)-[::-30]-[::60]-[::-60]\textcolor{red}{{\textbf N}}(-[::-60,0.75,,,red,thick]\textcolor{red}{{\textbf H}})-[::60,,,,,red,thick](=[::60,0.75,,,red]\textcolor{red}{{\textbf O}})
}

% Allylamin zu AFM-Spitze
\definesubmol\AllylaminSP{
	(=[::60,0.75,,,red]\textcolor{red}{{\textbf O}})-[::-60,,,,red,thick]\textcolor{red}{{\textbf N}}(-[::-60,0.75,,,red,thick]\textcolor{red}{{\textbf H}})-[::60]-[::-60]-[::60]C(-[::60,,,,thick]C-[::0,,,,thick]C)(-[::-120,,,,thick]C-[::0,,,,thick]C)
}

\vspace{3cm}

%% Aufbau

\begin{figure}[H]
	\begin{flushleft}
		\scalebox{1}{\begin{minipage}{\textwidth}
				\setpolymerdelim[]
				\setchemfig{atom style={rotate=0,scale=0.5}}
				\chemfig{[:0]!\AllylaminOF-[::-60]!\CMAVerbindung-[@{leftOut,0.5}::-100]O-[::60]-[::-30,0.001,,,,draw=none]!\CMAr-[::60]O-[::-60]-[::30,0.001,,,,draw=none]!\CMA-[@{rightOut,0.5}::-100]O-[::60]-[::-30,0.001,,,draw=none]!\CMArVerbindung-[::60]!\AllylaminSP}
				%\makebraces[45pt,30pt]{\!n}{left}{right}
				\makebraces[30pt,50pt]{\!n}{leftOut}{rightOut}
			\end{minipage}
		} % scalebox
	\end{flushleft}
\end{figure}


\end{document}