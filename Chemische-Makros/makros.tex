%===========================
%%% Makros für NHS/EDC-Chemie 
%===========================

% EDC
\newcommand{\EDC}{	
	\chemfig{H_3C-[:30,.75]-[:-30,.75]N=[:30,.75]C=[:30,.75]N-[:-30,.75]-[:30,.75]-[:-30,.75]-[:30,.75]\chembelow{N\rlap{${}^+$}}{H}(-[:90]CH_3)(-[:-30]CH_3)
		(-[:30,0.6,,,draw=none]Cl\rlap{${}^-$})}
}

% Isourea
\newcommand{\OAcylisourea}{
	\chemfig{
		[:90,.75]H_3C-[:30]-[:-30]N-[:30](-O-[:30](-\textcolor{red}{\textbf R_1})=[:-30]O)=[:-30]N-[:30]-[:-30]-[:30]-[:-30]\chembelow{N\rlap{${}^+$}}{H}(-[:90]CH_3)(-[:-30]CH_3)
	}
}

% Rest für den NHS-Ester
\definesubmol{REster}{\textcolor{red}{\textbf R_1}-[::30,.75](=[::60,.75]O)-[::-60,.75]O}

% NHS-Ester
\newcommand{\Ester}{
	\chemfig{
		[,.75]!{REster}-[:30,1]N*5(-(=O)---(=O)-)
	}	
}

% NHS
\newcommand{\NHS}{
	\chemfig{
		[:30,.75]HO-[,1]N*5(-(=O)---(=O)-)
	}
}

% Amidbindung
\newcommand{\Amidbindung}{
	\chemfig{
		[:30,1]\textcolor{red}{\textbf R_1}-(=[:90,.75]O)-[:-30]N(-[:-90,.75]H)-\textcolor{red}{\textbf R_2}
	}	
}

% Primäres Amin
\newcommand{\PrimAmin}{
	\chemfig{
		[,1.25]H_2N-\textcolor{red}{\textbf R_2}
	}	
}

% Rest mit Carbonsäure
\newcommand{\Rest}{
	\chemfig{
		[,.75]\textcolor{red}{\textbf R_1}-[:30](=[:90]O)-[:-30]OH
	}
}

%===========================
%%% Makros für die Amidhydrolyse
%===========================

% Wasser
\newcommand{\Wasser}{
	\setchemfig{angle increment = 30}
	\chemfig{H-[1,0.75]@{n1}\lewis{13,O}-[-1,0.75]H}
}

% OH-Ion
\newcommand{\OHIon}{
	\setchemfig{angle increment = 30}
	\chemfig{
		H@{oh}\chemabove{\lewis{026,O}}{\hspace{12pt}\rlap{-}}
	}
}

% Amidbindung allgemein (Amidbindung)
\renewcommand{\Amidbindung}{
	\chemfig{
		[:30,1]\textcolor{red}{\textbf R_1}-@{n2}(=[:90,.75]O)-[:-30]\chembelow{N}{H}-\textcolor{red}{\textbf R_2}
	}	
}

% Amidbindung mit addiertem Wasser und protoniert im neutralen (AmidbindungII)
\newcommand{\AmidbindungII}{
	\setchemfig{angle increment = 30}
	\chemfig{
		\textcolor{red}{\textbf R_1}-[1](-[-3.5,0.75]HO)(-[@{n3}3,0.75]@{n4}\lewis{024,O})-[@{n5}-1]@{n6}\chembelow{N}{\hspace{-5pt}\rlap{+}}H_2-[1]\textcolor{red}{\textbf R_2}
	}
}

% protonierte Amidbindung (AmidbindungIII)
\newcommand{\AmidbindungIII}{
	\setchemfig{angle increment = 30}
	\chemfig{
		\textcolor{red}{\textbf R_1}-[1](-[3,0.75]\lewis{24,O}H)(-[-3,0.2,,,draw = none]@{pa1}\hspace{-5pt}\rlap{+})-[-1]\chembelow{N}{H}-[1]\textcolor{red}{\textbf R_2}
	}
}

% Amidbindung mit addiertem Wasser im sauren (AmidbindungIV)
\newcommand{\AmidbindungIV}{
	\setchemfig{angle increment = 30}
	\chemfig{
		\textcolor{red}{\textbf R_1}-[1](-[3,0.75]\lewis{24,O}H)(-[-3.5,0.75]HO)-[-1]\chembelow{N}{H}-[1]\textcolor{red}{\textbf R_2}
	}
}

% Amidbindung-Wasser-addukt im sauren mit protoniertem Stickstoff (AmidbindungV)
\newcommand{\AmidbindungV}{
	\setchemfig{angle increment = 30}
	\chemfig{
		\textcolor{red}{\textbf R_1}-[1](-[@{av1}3,0.75]\lewis{24,O}-[@{av2},0.75]@{av3}H)(-[-3.5,0.75]HO)-[@{av4}-1]\chembelow{N}{\hspace{-5pt}\rlap{+}}H_2-[1]\textcolor{red}{\textbf R_2}
	}
}

% Amidbindung-OH-addukt im basischen (AmidbindungVI)
\newcommand{\AmidbindungVI}{
	\setchemfig{angle increment = 30}
	\chemfig{
		\textcolor{red}{\textbf R_1}-[1](-[3,0.75]\chemabove{\lewis{024,O}}{\hspace{10pt}\rlap{-}})(-[-3.5,0.75]HO)-[-1]\chembelow{N}{H}-[1]\textcolor{red}{\textbf R_2}
	}
}

% Amidbindung-OH-addukt mit protoniertem Stickstoff im basischen (AmibdingungVII)
\newcommand{\AmidbindungVII}{
	\setchemfig{angle increment = 30}
	\chemfig{
		\textcolor{red}{\textbf R_1}-[1](-[@{aoh1}3,0.75]@{aoh2}\chemabove{\lewis{024,O}}{\hspace{10pt}\rlap{-}})(-[-3.5,0.75]HO)-[@{aoh3}-1]@{aoh4}\chembelow{N}{\hspace{-5pt}\rlap{+}}H_2-[1]\textcolor{red}{\textbf R_2}
	}
}

% Carbonsäure
\newcommand{\Acid}{
	\setchemfig{angle increment = 30}
	\chemfig{
		\textcolor{red}{\textbf R_1}-[1](=[3,0.75]O)-[-1]OH
	}
}

% primäres Amin
\renewcommand{\PrimAmin}{
	\setchemfig{angle increment = 30}
	\chemfig{
		H_2\lewis{2,N}-\textcolor{red}{\textbf R_2}
	}
}

% protoniertes Amin
\newcommand{\PrimAminII}{
	\setchemfig{angle increment = 30}
	\chemfig{
		H_3\chemabove{N}{\hspace{-5pt}\rlap{+}}-\textcolor{red}{\textbf R_2}
	}
}

%===========================================
%%% Makros für den molekularen Aufbau der Experimente
%===========================================

% Polymerklammern (für Details siehe Dokumentation des Chemfig Paketes)
\newcommand\setpolymerdelim[2]{\def\delimleft{#1}\def\delimright{#2}}
\def\makebraces[#1,#2]#3#4#5{%
	\edef\delimhalfdim{\the\dimexpr(#1+#2)/2}%
	\edef\delimvshift{\the\dimexpr(#1-#2)/2}%
	\chemmove{%
		\node[at=(#4),yshift=(\delimvshift)]{$\left\delimleft\vrule height\delimhalfdim depth\delimhalfdim width0pt\right.$};
		\node[at=(#5),yshift=(\delimvshift)]{$\left.\vrule height\delimhalfdim depth\delimhalfdim width0pt\right\delimright_{\rlap{$\scriptstyle#3$}}$};
	} % \chemmove
} % \def\delimleft\makebraces

% Beispiel: Polyethylen
%
% \setpolymerdelim[]
% \chemfig{\vphantom{CH_2}-[@{op,.75}]CH_2-CH_2-[@{cl,0.25}]} 
% \makebraces[5pt,5pt]{\!n}{op}{cl}

% Diese Befehle sorgen dafür dass funktionelle Gruppen immer am O bzw. C angesetzt werden
\definesubmol\OH[HO]{OH}
\definesubmol\OR[RO]{OR}

% CMA moleküle (\CMA - normales CMA und \CMAr - rotieretes CMA)
\definesubmol\CMA{
	(-[::70,1.5](-[::140,0.75]!\OH)-[::-80,1.75](-[::130,0.75]!\OR)-[::40,1.2])-[::30](-[::-90,0.75]-[::-30,0.75]!\OR)-[::-40,1.75]O-[::80,1.5]
}


\definesubmol\CMAVerbindung{
	O-[::60]-[::30,0.75](-[::150](-[::-60]...)-[::-140,1.5](-[::140,0.75]!\OH)-[::-80,1.75](-[::130,0.75]!\OR)-[::40,1.2])-[::-70,1.75]O-[::80,1.5]
}

\definesubmol\CMAr{
	(-[::-30](-[::90,0.75]-[::30,0.75]!\OR-[@{left,0.25}::-30,,,,draw=none])-[::40,1.75]O-[::-80,1.5])-[::-70,1.5](-[::240,0.75]!\OH)-[::80,1.75](-[::-120,0.75]!\OR-[@{right,0.25}::10,,,,draw=none])-[::-40,1.2]
}

\definesubmol\CMArVerbindung{
	(-[::-30](-[::90,0.75]!\OR)-[::40,1.75]O-[::-80,1.5])-[::-70,1.5](-[::240,0.75]!\OH)-[::80,1.75](-[::-40,1.2]-[::60]...)-[::-70,0.75]-[::30]O
}

% Allylamin Reste

%DLC-Substrat zu Allylamin
\definesubmol\AllylaminOF{
	C(-[::90,,,,thick]C-[::0,,,,thick]C)(-[::-90,,,,thick]C-[::0,,,,thick]C)-[::-30]-[::60]-[::-60]\textcolor{red}{{\textbf N}}(-[::-60,0.75,,,red,thick]\textcolor{red}{{\textbf H}})-[::60,,,,,red,thick](=[::60,0.75,,,red]\textcolor{red}{{\textbf O}})
}

% Allylamin zu AFM-Spitze
\definesubmol\AllylaminSP{
	(=[::60,0.75,,,red]\textcolor{red}{{\textbf O}})-[::-60,,,,red,thick]\textcolor{red}{{\textbf N}}(-[::-60,0.75,,,red,thick]\textcolor{red}{{\textbf H}})-[::60]-[::-60]-[::60]C(-[::60,,,,thick]C-[::0,,,,thick]C)(-[::-120,,,,thick]C-[::0,,,,thick]C)
}
